\documentclass{article}
\usepackage{graphicx}
\usepackage{amsmath}
\usepackage{esint}
\usepackage{verbatim}

\begin{document}
\section{Easy}
\begin{equation}
\textrm{Please type me! The quick brown fox jumps over the lazy dog.}
\end{equation}

\begin{equation}
e^{i\pi}+1=0
\end{equation}

\begin{equation}
e^{i\theta}=\cos\theta+i\sin\theta
\end{equation}

\begin{equation}
G_{\mu\nu}+\Delta g_{\nu\mu}= {\frac{{8\pi G}}{c^4}}T_{\mu\nu}
\end{equation}

\begin{equation}
x=\frac{-b\pm\sqrt{b^2-4ac}}{2a}
\end{equation}

\begin{equation}
\overrightarrow{L}=\overrightarrow{r}\times\overrightarrow{p}
\end{equation}

\begin{equation}
\sqrt[3]{2}
\end{equation}

\begin{equation}
(x+y)^n=\sum_{r=0}^n\binom{n}{r}x^ry^{n-r}
\end{equation}

\begin{equation}
\sqrt{\frac{a_{1}^{2}+\cdots+a_{n}^2}{n}}\geq{\frac{a_{1}+\cdots+a_{n}}{n}}\geq\sqrt[n]{a_{1}\cdots a_{n}}\geq \frac{n}{{\frac{1}{a_{1}}+\cdots+\frac{1}{a_{n}}}}
\end{equation}

\begin{equation}
{\vert{\langle{x,y}\rangle}\vert}^{2}\leq\langle{x,x}\rangle\cdot \langle{y,y}\rangle
\end{equation}

\begin{equation}
\begin{aligned}
A1:& \ \varphi \longrightarrow (\psi\longrightarrow\varphi) \\
A2:& \ (\varphi\rightarrow(\psi\rightarrow\theta)) \longrightarrow((\varphi \rightarrow \psi) \rightarrow (\varphi \rightarrow \theta)) \\
A3:& \ (\neg \varphi \rightarrow \neg \psi) \longrightarrow (\psi \rightarrow \varphi)
\end{aligned}
\end{equation}

\newpage
\section{Medium}

\begin{equation}
1_A=
\begin{cases}
1 & \text{if} \ x\in A \\
0 & \text{if} \ x\notin A
\end{cases}
\end{equation}

\begin{equation}
n \ \underbrace{\uparrow \ \cdots \ \uparrow}_{n}n=n \rightarrow n \rightarrow n
\end{equation}
In the following, note the spacing between the $=$ and the ${}^{1}1$, ${}^{2}2$, and ${{}^{{}^{3}3}3}$.

\begin{equation*}
\begin{aligned}
1 \ \uparrow \ 1 &= {}^{1}1=1\\
2 \ \uparrow\uparrow 2&={}^{2}2=4
\end{aligned}
\end{equation*}

\begin{equation}
3\uparrow\uparrow\uparrow3={{}^{{}^{3}3}3}=3\uparrow\uparrow 3 \uparrow\uparrow 3=\underbrace{3^{3^{3^{3^{3^{3^{{{\cdot}^{\cdot^{\cdot^3}}}}}}}}}}_{{3^{3^{3}}} \ \text{threes}}
\end{equation}

\begin{equation}
\frac{d}{dx}f(x)=\underset{\Delta x\rightarrow 0}\lim\frac{f(x+\Delta x)-f(x)}{\Delta x}
\end{equation}

\begin{equation}
H_2O(l)+H_2O(l)\rightleftharpoons H_3O^+(aq)+OH^-(aq)
\end{equation}

\begin{equation}
\Gamma(n+1)\overset{\text{def}}=\int_0^\infty e^{-t}t^n \ dt
\end{equation}

\begin{equation}
gcd(n,m \ \text{mod }  n); \ \ \ x\equiv y \ (\text{mod} \ b); \ \ \ x\equiv y \ \ \text{mod} \ c; \ \ \ x\equiv y \ \ (d)
\end{equation}
In the following, note the bold symbols.

\begin{equation*}
\begin{aligned}
\nabla\cdot \mathbf{E}&=\frac{\rho}{\varepsilon_0} \\
\nabla\cdot \mathbf{B}&=0\\
\nabla\times \mathbf{E}&=-\frac{\partial \mathbf{B}}{\partial t}
\end{aligned}
\end{equation*}
\begin{equation}
\nabla \times \mathbf{B}
=\mu_0\mathbf{J}+\mu_0\varepsilon_0\frac{\partial \mathbf{E}}{\partial t}
\end{equation}
For the following exercise, you will need to \verb|\usepackage{esint}| to get the symbol $\iint$.

\begin{equation*}
\begin{aligned}
\oiint_{\partial V} \mathbf{E}\cdot d\mathbf{A}&=\frac{Q(V)}{\varepsilon_0} \\
\oiint_{\partial V} \mathbf{B}\cdot d\mathbf{A}&=0 \\
\oint_{\partial S} \mathbf{E}\cdot dl&=-\frac{\partial\Phi_{B,S}}{\partial t}
\end{aligned}
\end{equation*}
\begin{equation}
\oint_{\partial S} \mathbf{B}\cdot dl=\mu_0I_S+\mu_0\varepsilon_0\frac{\partial\Phi_{E,S}}{\partial t}
\end{equation}

\end{document}

